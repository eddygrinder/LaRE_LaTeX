\chapter{Conclusões}
\label{Capítulo6}
\begin{flushright}
\textit{``Make your lives extraordinary''} \\[0.5em]
--- Professor John Keating, \textit{Dead Poets Society}
\end{flushright}
O principal objectivo desta dissertação foi o desenvolvimento de um \acrshort{laboratório remoto} para o ensino da Electrónica, concebido como uma alternativa viável ao \acrshort{visir}, com o compromisso adicional de constituir um projecto \textit{open-source}, acessível a qualquer instituição de ensino, independentemente da sua localização geográfica. Este objectivo foi plenamente atingido, tendo sido concretizada a construção da matriz de placas \acrshort{lare}, elemento central do sistema, cuja especificação técnica se encontra detalhada no \textit{datasheet} apresentado no Anexo~\ref{AppendixA}. Esta matriz permite o controlo remoto das experiências com base numa arquitectura controlada por \textit{software} e sem dependência de software proprietário.

Um segundo objectivo consistiu em garantir que o \acrshort{lare} pudesse ser utilizado sem dependência de licenças pagas, ultrapassando as limitações impostas por soluções existentes como o \acrshort{visir}, cujo funcionamento depende do \acrshort{labview} — um \textit{software} proprietário, conforme discutido na Secção~\ref{sec:contextualização}, que restringe o acesso a instituições sem capacidade para adquirir as respectivas licenças. Este objectivo foi igualmente cumprido, mediante a adopção de soluções baseadas em software livre e plataformas de \textit{hardware} livres. Promove-se, dessa forma, o acesso universal e equitativo a recursos de ensino remoto de Electrónica, contribuindo para a inclusão de instituições com recursos mais limitados e reforçando o carácter aberto e expansível do sistema desenvolvido. Pretendeu-se assegurar, também, que todo o sistema pudesse ser executado em plataformas livres e de baixo custo, nomeadamente no \textit{Raspberry Pi}, com recurso exclusivo a sistemas operativos baseados em \textit{Linux}. Este objectivo não foi integralmente cumprido, não por insuficiência do desenvolvimento realizado, mas devido à incompatibilidade da biblioteca \textit{pyVirtualBench} com arquitecturas \acrshort{arm} e ambientes \textit{Linux}, o que inviabilizou a sua execução directa no \textit{Raspberry Pi}. Como solução alternativa, foi adoptada uma arquitectura em que um portátil com sistema \textit{Windows} e suporte para a referida biblioteca estabelece a ligação com \acrshort{virtualbench}, processando os dados e transmitindo-os ao \textit{Raspberry Pi}, que, por sua vez, gere a comunicação com o \acrshort{lare}. Apesar deste desvio da intenção inicial, a solução revelou-se funcional e mantém os princípios de modularidade e interoperabilidade do sistema, podendo ser ajustada futuramente com a evolução da compatibilidade entre bibliotecas e plataformas livres.

Por fim, procurou-se assegurar que o \acrshort{lare} fosse um \acrshort{laboratório remoto} expansível e adaptável a diferentes contextos pedagógicos. Este objectivo foi parcialmente \textbf{Opinião PROF} atingido: embora a arquitectura modular desenvolvida permita a futura integração de novos módulos e funcionalidades, a actual implementação cobre apenas um conjunto básico de experiências. No entanto, o desenho aberto do sistema e a documentação produzida oferecem condições para que futuras versões sejam estendidas com facilidade.

Neste contexto, o sistema desenvolvido representa um primeiro passo funcional nesta direcção, configurando-se como um protótipo plenamente operativo, mas que ainda oferece amplo espaço para aperfeiçoamentos e evolução futura. A sua arquitectura modular e o carácter aberto do código permitem que seja expandido, adaptado a diferentes contextos pedagógicos e enriquecido com novas funcionalidades ao longo do tempo. Nesse sentido, o trabalho aqui apresentado constitui uma base sólida e adaptável, que poderá servir de ponto de partida para o desenvolvimento de futuras soluções no domínio dos laboratórios remotos.

\section{Limitações e trabalho futuro}
\label{limitacoes}
Embora o sistema desenvolvido tenha atingido os principais objectivos propostos, algumas limitações e desafios foram identificados ao longo do processo. A principal limitação reside na incompatibilidade da biblioteca \textit{pyVirtualBench}, que, como referido no capítulo anterior, impede a execução completa do sistema em plataformas como o \textit{Raspberry Pi}. Esta restrição compromete a portabilidade e a flexibilidade do \acrshort{lare}. Assim, um passo fundamental rumo à total liberdade e gratuitidade do projecto passaria pelo desenvolvimento da compatibilidade da \textit{pyVirtualBench} com arquitecturas \acrshort{arm} e sistemas operativos baseados em \textit{Linux}.

Apesar de a arquitectura modular permitir a futura adição de novos módulos e experiências, a actual implementação abrange apenas um conjunto básico. A expansibilidade do sistema está assegurada, mas levanta questões de ordem \textit{software} e/ou \textit{hardware}. Tal como está concebido, a \textit{string} de controlo utilizada para activar os relés varia consoante a experiência: no caso da Lei de \textit{Ohm}, é composta por 8 \textit{bits}; nas experiências de rectificação e filtragem, são necessários 13 \textit{bits}, correspondendo ao número de relés a comandar. No \acrshort{lare}, optou-se por utilizar um conjunto de cinco pinos de controlo dedicados por placa, como apresentado na Tabela~\ref{Table:funcSN74HC595}.

No entanto, a adição de novas experiências, ou o aumento do número de relés por experiência, exigiria o ajuste da \textit{string} de controlo, com o risco de esta se tornar demasiado longa e a transmissão de dados excessivamente lenta. Para contornar esta limitação sem modificações de fundo no \textit{hardware}, uma abordagem possível seria acrescentar duas ligações físicas à placa do \textit{Raspberry Pi}, activando dois pinos adicionais dedicados ao $\overline{OE}$ e ao $SER$, ambos já disponíveis. Esta actualização seria relativamente simples em termos de programação. Numa perspectiva mais abrangente, os esquemas e as placas poderiam ser redesenhados para permitir a expansão modular do sistema, sem necessidade de cablagens adicionais. Por exemplo, todos os pinos dos registos poderiam permanecer partilhados, com excepção do $\overline{OE}$, que seria ligado ao pino respectivo através de um \textit{jumper}\footnote{Poderia colocar-se a questão de um único pino do \textit{Raspberry Pi} alimentar simultaneamente vários registos de deslocamento, uma vez que as saídas destes são controladas através do pino $\overline{OE}$. No entanto, os circuitos integrados \textit{74HC595} são destinados a aplicações \textit{CMOS} e, em condições normais de operação, apresentam uma corrente de entrada típica da ordem de apenas \SI{1}{\micro\ampere}\cite{SN74HC595}, o que torna este tipo de ligação perfeitamente segura do ponto de vista elétrico.}. 

Neste protótipo, o \acrshort{ide} foi implementado com o objetivo de ser o mais leve, simples e direto possível. No entanto, há ainda espaço para atualizações e melhorias que proporcionem uma experiência ainda mais fluida, dinâmica e amigável. Nesta fase do desenvolvimento, o utilizador não tem informação sobre o estado dos gráficos; por isso, a implementação de um sistema de \textit{feedback} visual poderia melhorar significativamente a experiência do utilizador. Além disso, a adição de funcionalidades como a possibilidade de guardar os gráficos gerados e a exportação dos dados recolhidos para formatos como \textit{CSV} ou \textit{Excel} poderia aumentar a utilidade do \acrshort{ide}.

Outra melhoria prende-se com o \textit{Flask}, que, apesar de ser uma solução leve e eficaz para o desenvolvimento de aplicações \textit{web}, pode não ser a mais adequada para projetos de maiores dimensões e que requeiram mais funcionalidades \cite{FlaskvsDjango, Djangovsflask}. Sendo este o caso, deveria ser considerado um estudo sobre o \textit{Django} ou mesmo o \textit{FastAPI}. Para além da escolha do \textit{framework}, poderá também ser vantajosa a implementação de um sistema de autenticação e gestão de utilizadores, permitindo que diferentes alunos ou utilizadores acedam ao \acrshort{lare} com as suas próprias credenciais, aumentando, desta forma, a segurança do sistema.
