\chapter{Conclusões}
\label{Capítulo6}
\begin{flushright}
\textit{``Make your lives extraordinary''} \\[0.5em]
--- Professor John Keating, \textit{Dead Poets Society}
\end{flushright}
O principal objectivo desta dissertação foi o desenvolvimento de um \acrshort{laboratório remoto} para o ensino da Electrónica, concebido como uma alternativa viável ao \acrshort{visir}, com o compromisso adicional de constituir um projecto \textit{open-source}, acessível a qualquer instituição de ensino, independentemente da sua localização geográfica. Este objectivo foi plenamente atingido, tendo sido concretizada a construção da matriz de placas \acrshort{lare}, elemento central do sistema, cuja especificação técnica se encontra detalhada no \textit{datasheet} apresentado no Anexo~\ref{AppendixA}. Esta matriz permite o controlo remoto das experiências com base numa arquitectura controlada por \textit{software} e sem dependência de software proprietário.

Um segundo objectivo consistiu em garantir que o \acrshort{lare} pudesse ser utilizado sem dependência de licenças pagas, ultrapassando as limitações impostas por soluções existentes como o \acrshort{visir}, cujo funcionamento depende do \acrshort{labview} — um \textit{software} proprietário, conforme discutido na Secção~\ref{sec:contextualização}, que restringe o acesso a instituições sem capacidade para adquirir as respectivas licenças. Este objectivo foi igualmente cumprido, mediante a adopção de soluções baseadas em software livre e plataformas de \textit{hardware} livres. Promove-se, dessa forma, o acesso universal e equitativo a recursos de ensino remoto de Electrónica, contribuindo para a inclusão de instituições com recursos mais limitados e reforçando o carácter aberto e expansível do sistema desenvolvido. Pretendeu-se assegurar, também, que todo o sistema pudesse ser executado em plataformas livres e de baixo custo, nomeadamente no \textit{Raspberry Pi}, com recurso exclusivo a sistemas operativos baseados em \textit{Linux}. Este objectivo não foi integralmente cumprido, não por insuficiência do desenvolvimento realizado, mas devido à incompatibilidade da biblioteca \textit{pyVirtualBench} com arquitecturas \acrshort{arm} e ambientes \textit{Linux}, o que inviabilizou a sua execução directa no \textit{Raspberry Pi}. Como solução alternativa, foi adoptada uma arquitectura em que um portátil com sistema \textit{Windows} e suporte para a referida biblioteca estabelece a ligação com \acrshort{virtualbench}, processando os dados e transmitindo-os ao \textit{Raspberry Pi}, que, por sua vez, gere a comunicação com o \acrshort{lare}. Apesar deste desvio da intenção inicial, a solução revelou-se funcional e mantém os princípios de modularidade e interoperabilidade do sistema, podendo ser ajustada futuramente com a evolução da compatibilidade entre bibliotecas e plataformas livres.

Por fim, procurou-se assegurar que o \acrshort{lare} fosse um \acrshort{laboratório remoto} expansível e adaptável a diferentes contextos pedagógicos. Este objectivo foi parcialmente \textbf{Opinião PROF} atingido: embora a arquitectura modular desenvolvida permita a futura integração de novos módulos e funcionalidades, a actual implementação cobre apenas um conjunto básico de experiências. No entanto, o desenho aberto do sistema e a documentação produzida oferecem condições para que futuras versões sejam estendidas com facilidade.

Neste contexto, o sistema desenvolvido representa um primeiro passo funcional nesta direcção, configurando-se como um protótipo plenamente operativo, mas que ainda oferece amplo espaço para aperfeiçoamentos e evolução futura. A sua arquitectura modular e o carácter aberto do código permitem que seja expandido, adaptado a diferentes contextos pedagógicos e enriquecido com novas funcionalidades ao longo do tempo. Nesse sentido, o trabalho aqui apresentado constitui uma base sólida e adaptável, que poderá servir de ponto de partida para o desenvolvimento de futuras soluções no domínio dos laboratórios remotos.

\section{Limitações}
\label{limitacoes}
Embora o sistema desenvolvido tenha alcançado os principais objectivos propostos, algumas limitações e desafios foram identificados durante o processo. A principal limitação reside na incompatibilidade da biblioteca \textit{pyVirtualBench} com arquitecturas \acrshort{arm} e sistemas operativos \textit{Linux}. Esta dependência impede a execução completa do sistema em plataformas como o \textit{Raspberry Pi}, limitando a portabilidade e a flexibilidade do \acrshort{lare}. Este projecto não contempla a figura de um administrador do sistema, o que pode levar a dificuldades na gestão de utilizadores e na manutenção do ambiente de trabalho. A ausência de um administrador dedicado pode resultar em desafios na configuração e no suporte técnico, especialmente em ambientes educativos com múltiplos utilizadores. \textbf{Forçado? Opinião PROF} O \acrshort{ide}, no entanto, necessita de ser mais robusto, já que, por exemplo, 

\section{Trabalho Futuro}
\label{trabalho_futuro}
blá