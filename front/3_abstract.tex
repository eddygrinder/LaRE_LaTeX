%----------------------------------------------------------------------------------------
%	ABSTRACT PAGES
%----------------------------------------------------------------------------------------

% IMPORTANT NOTE: the abstract must always be written in two languages. If the dissertation
% is written in Portuguese you have selected 'portuguese' as the language in the document class.
% Therefore, the portuguese version of the abstract must come first, so write it in the
% below area denoted by 'MAIN LANGUAGE ABSTRACT'. The english version follows in the
% 'SECOND LANGUAGE ABSTRACT' section.
% If the dissertation is written in English, first will come the abstract in English
% ('MAIN LANGUAGE ABSTRACT') and then in Portuguese ('SECOND LANGUAGE ABSTRACT').

\begin{abstract}
%%%%%%%%%%%%%%%%%%%%%%%%%%%%%% MAIN LANGUAGE ABSTRACT %%%%%%%%%%%%%%%%%%%%%%%%%%%%%%%%%%

Aqui deverá ser apresentado o resumo de todo o trabalho efetuado. Esta secção não deverá exceder uma página.

Deve contextualizar o problema que pretende resolver ou a hipótese que irá formular, procure evidenciar as vantagens e desvantagens (se as houver) da solução encontrada, como também a forma através da qual a solução/hipótese foi validada. Neste último ponto, deverá referir-se aos desenvolvimentos efetuados, e à forma como validou (conformidade) e avaliou (desempenho) a solução encontrada.

A dissertação deve conter sempre duas versões do resumo: uma primeira no idioma do texto principal e a segunda num outro idioma. Este \textit{template} assume que os dois idiomas em consideração são sempre Português e Inglês, assim, a classe irá colocar os cabeçalhos respetivos de acordo com o idioma selecionado nas opções da classe no ficheiro \file{main.tex}. 

%----------------------------------------------------------------------------------------

\vspace*{10mm} 
\noindent
\textbf{\keywordslabel}: Lista, separada por vírgulas, de palavras, frases, ou acrónimos chave no âmbito do trabalho descrito neste texto. 

%%%%%%%%%%%%%%%%%%%%%%%%% END OF THE MAIN LANGUAGE ABSTRACT %%%%%%%%%%%%%%%%%%%%%%%%%%%%%%
\end{abstract}
\begin{secondlangabstract}
%%%%%%%%%%%%%%%%%%%%%%%%%%%%%% SECOND LANGUAGE ABSTRACT %%%%%%%%%%%%%%%%%%%%%%%%%%%%%%%%%%

The summary of all the developed work should be presented here. This section should not exceed one page.

Start the abstract with the contextualization of the problem you intend to solve or the hypothesis you will formulate. Try to highlight the advantages and disadvantages (if any) of the solution found, as well as the way in which the solution/hypothesis was validated. In this last point, you should refer to the developments made, and to the way you validated (compliance) and evaluated (performance) the solution found.

The dissertation must always contain two versions of the abstract: a first in the language of the main text and the second one in another language. This template assumes that the two languages are always Portuguese and English, therefore, the class will place the correct section headers according to the language selected in the class options in the \file{main.tex} file.


%----------------------------------------------------------------------------------------

\vspace*{10mm} 
\noindent
\textbf{\keywordslabel}: Comma separated list of words, phrases, or key acronyms within the scope of your developed work. 

%%%%%%%%%%%%%%%%%%%%%%%%%% END OF THE SECOND LANGUAGE ABSTRACT %%%%%%%%%%%%%%%%%%%%%%%%%%%%%
\end{secondlangabstract}

