%----------------------------------------------------------------------------------------
%	GLOSSARY
%----------------------------------------------------------------------------------------

% Only the used entries will be displayed in the printed list, ie, you need to used a term at least once
% In italic if not in the main document language

% terms definition usage:
% \newglossaryentry{<tag>}{name={<term>},description={<description of the term>}}

\newglossaryentry{w3c}{
    name=W3C,
    description={O \acrfull{W3c} é uma organização internacional que desenvolve padrões e diretrizes para a \textit{web} visando garantir que esta permaneça aberta, acessível e interoperável para todos. Os principais objetivos são promover a compatibilidade entre diferentes sistemas e garantir que a \textit{web} seja acessível a todos os utilizadores, independentemente das suas capacidades ou dispositivos. O \acrshort{W3c} trabalha em várias áreas, incluindo \acrfull{html} e \acrfull{css} \cite{W3C}.}
}

\newglossaryentry{sbc}{
    name=SBC,
    description={\acrfull{Sbc} é um computador completo integrado numa única placa de circuito impresso, que inclui processador, memória, interfaces de entrada/saída e, muitas vezes, conectividade de rede. Os \acrshort{sbc}, como o Raspberry PI, são amplamente utilizados em aplicações de ensino, automação, prototipagem e sistemas embarcados, devido ao seu baixo custo, tamanho reduzido e versatilidade.}
}

\newglossaryentry{eeprom}{
    name=EEPROM,
    description={\acrfull{Eeprom} é um tipo de memória não volátil que pode ser escrita e apagada electricamente. É utilizada para armazenar pequenas quantidades de dados que devem ser preservados mesmo após desligar o sistema, como configurações, identificadores de \textit{hardware} ou parâmetros de funcionamento.}
}

\newglossaryentry{hat}{
    name=HAT,
    description={O termo \acrfull{Hat} refere-se a placas de expansão desenvolvidas segundo um padrão específico para o Raspberry PI. Estas placas ligam-se diretamente ao \acrfull{gpio} e incluem uma \acrshort{Eeprom} que permite ao sistema detectar automaticamente o acessório e configurar os respectivos pinos e interfaces de forma transparente.}
}

\newglossaryentry{kicad}{
    name=KiCad,
    description={O KiCad é um conjunto de ferramentas de \textit{software} de código aberto dedicada à Automatização do \textit{design} electrónico \acrfull{eda}. Esta ferramenta integra funcionalidades para a captura de esquemas elétricos e o desenho de placas de circuito impresso (\acrfull{pcb}), permitindo a exportação dos projetos nos formatos Gerber e IPC-2581. O KiCad é compatível com os sistemas operativos Windows, Linux e MacOS, sendo distribuída sob a licença GNU General Public License, versão 3 (GPL v3)}
}


\newglossaryentry{python}{
    name=Python,
    description={O Python é uma linguagem de programação de alto nível, orientada a objecto, ou seja, com sintaxe mais simplificada e próxima da linguagem humana. É utilizada em \textit{web} ou servidores. Tem uma biblioteca padrão bastante completa e uma enorme comunidade que desenvolve ferramentas e bibliotecas (\textit{frameworks}) adicionais. O código é aberto e gratuito \cite{ThePython}. O Python usa a indentação para delimitar blocos de código. Como curiosidade esta linguagem foi baptizada com este nome pelo seu criador, Guido van Rossum, como referência aos Monty Python.}
}

\newglossaryentry{PIP}{
    name=pip,
    description={O \acrfull{pip} é o instalador de pacotes para Python que é usado para instalar pacotes dos repositórios Python e outros repositórios.}
}

\newglossaryentry{API}{
    name=API,
    description={Uma \acrfull{API} é um conjunto de definições e protocolos que permite que diferentes softwares se comuniquem entre si. No contexto de programação, uma \acrshort{API} fornece uma \textit{interface} através da qual se pode interagir com uma aplicação, biblioteca, ou serviço, utilizando comandos, funções, ou rotinas específicas.
        }}

\newglossaryentry{RaspberryPI}{
    name=Raspberry PI,
    description={Raspberry PI é uma gama de pequenos computadores acessíveis e versáteis que podem ser utilizados para vários projectos e fins, desenvolvidos pela Raspberry PI Foundation, uma organização educacional no Reino Unido. Esses dispositivos foram projetados para promover o ensino de ciências da computação básicas em escolas e países em desenvolvimento, além de serem populares entre entusiastas de eletrónica e \textit{makers} devido à sua versatilidade e custo acessível. Desde o seu lançamento, os modelos de Raspberry PI evoluíram significativamente, oferecendo maiores capacidades de processamento, memória e conectividade, adequados a uma ampla gama de aplicações, desde simples projetos de bricolagem a sistemas embebidos complexos e até mesmo servidores de baixo custo \cite{pifoundation}\cite{Raspberrypi}.
        }
}

\newglossaryentry{ESP32}{
    name=ESP32,
    description={O ESP32 é um microcontrolador de baixo custo e baixa potência com suporte para rede sem fios e \textit{Bluetooth}, desenvolvido pela Espressif Systems. É amplamente utilizado em projectos de \acrfull{iot} devido às suas capacidades de conectividade e processamento.}
}

\newglossaryentry{arduino}{
    name=Arduino,
    description={O Arduino é uma placa de desenvolvimento de código aberto baseada em \textit{hardware} e \textit{software} fáceis de utilizar. As placas Arduino são capazes de ler entradas - luz num sensor, um dedo num botão ou uma mensagem do \textit{Twitter} - e transformá-las numa saída - activar um motor, ligar um \acrfull{LED}, publicar algo \textit{online} - utilizam a linguagem de programação Arduino (baseada em Wiring) e o \textit{software} Arduino (\acrfull{ide}), baseado em Processing \cite{arduino}.
        }}

\newglossaryentry{industria40}{
    name=Industry 4.0,
    description={A Indústria 4.0, uma iniciativa da Alemanha, converteu-se num termo adoptado globalmente na última década. Dez anos após a introdução da Indústria 4.0, a Comissão Europeia anunciou a Indústria 5.0. Considera-se que a Indústria 4.0 é orientada para a tecnologia, enquanto a Indústria 5.0 é orientada para o valor \cite{industry4.0}.}
}

\newglossaryentry{triac}{
    name=TRIAC,
    description={O \acrfull{Triac} é um tiristor constituído por quatro camadas \acrshort{npnp} ou \acrshort{pnpn}, com dois ânodos e uma porta. Contrariamente ao \acrshort{Scr}, o \acrshort{Triac} conduz nos dois sentidos da \acrshort{ca}, desde que se aplique à porta um impulso suficiente, positivo ou negativo. O \acrshort{Triac} é, geralmente, utilizado em \acrshort{ca}, no comando e controlo de potência} \cite{matias}.
}

\newglossaryentry{diac}{
    name=DIAC,
    description={O \acrfull{Diac} é um semicondutor que conduz a corrente eléctrica num só sentido, tal como o díodo, mas de forma controlada, isto é, só após um disparo na sua porta. O \acrshort{Diac} é constituído por quatro camadas \acrfull{npnp} ou \acrfull{pnpn}, tendo numa extremidade um ânodo e um cátodo e, ainda, a porta. Ao aplicar uma tensão positiva entre o ânodo e cátodo, ele conduz, desde que se aplique um impulso positivo na porta, com um determinado valor mínimo. Tanto o ânodo como a porta devem ser polarizados directamente, senão o \acrshort{Diac} não conduz} \cite{matias}.
}

\newglossaryentry{scr}{
    name=SCR,
    description={O \acrfull{Scr} é um semicondutor que conduz a corrente eléctrica nos dois sentidos, como se fossem dois díodos em paralelo e em sentidos contrários; isto é, ora funciona um díodo, durante a alternância positiva, ora funciona o outro na alternância negativa. A alimentação da porta do \acrshort{Triac} é, frequentemente, feita utilizando um \acrshort{Diac}} \cite{matias}.
}

\newglossaryentry{LABVIEW}{
    name=LabVIEW,
    description={O \textit{LabVIEW} é um ambiente de programação gráfica que proporciona aceleradores de produtividade únicos para o desenvolvimento de sistemas de teste, tais como uma abordagem intuitiva à programação, conetividade a qualquer instrumento e interfaces de utilizador totalmente integradas.}
}

%(Python Software Foundation, 2024; Beazley & Jones, 2013)

\newglossaryentry{pc/104}
{
    name=PC/104,
    description={As placas PC/104 são um padrão de computação embebida que se destaca pelo seu formato compacto e modular. Desenvolvidas pelo PC/104 Consortium, estas placas são amplamente utilizadas em aplicações industriais e de controlo, oferecendo uma combinação de robustez e flexibilidade.}
}

\newglossaryentry{JSON}
{
    name=JSON,
    description={\acrfull{json} é um formato leve para armazenamento e transporte de dados. É frequentemente utilizado na transmissão de informações entre um servidor e uma página \textit{web}, facilitando a comunicação entre diferentes sistemas. Além disso, o JSON é um formato auto-descritivo e de fácil compreensão, o que o torna amplamente adoptado no desenvolvimento de aplicações \textit{web} e na integração de \acrshort{api}s.}
}

\newglossaryentry{wrapper}{
    name=Wrapper,
    description={Um \textit{wrapper} em Python, neste contexto, refere-se a uma camada de código desenvolvida para encapsular e simplificar a utilização da interface de programação do VirtualBench. Essencialmente, os \textit{wrappers} funcionam como intermediários que permitem aos programadores interagir com a funcionalidade subjacente do VirtualBench através de comandos Python mais simples e diretos. Isso facilita a automação de tarefas e a integração do VirtualBench com outros componentes ou \textit{scripts} desenvolvidos em Python, proporcionando uma maneira mais eficiente e acessível de controlar e operar os dispositivos e funcionalidades oferecidos.}
}