%----------------------------------------------------------------------------------------
%	ACKNOWLEDGEMENTS
%----------------------------------------------------------------------------------------

\begin{acknowledgements}
``A ordem dos factores é arbitrária'', portanto, os agradecimentos seguem sem uma ordem específica.

Um agradecimento muito especial terá de ser dirigido ao meu orientador - ``Oh Captain, my Captain'' - o Professor André Vaz Fidalgo. Obrigado pela (muita) paciência, liberdade, tolerância, sapiência, orientação e apoio durante todo o processo de desenvolvimento deste trabalho.

José Saramago escreveu o seguinte: ``\textit{Filho é um ser que nos emprestaram para um curso intensivo de como amar alguém além de nós mesmos, de como mudar nossos piores defeitos para darmos os melhores exemplos e de aprendermos a ter coragem. Isto mesmo! Ser pai ou mãe é o maior ato de coragem que alguém pode ter, porque é se expor a todo tipo de dor, principalmente da incerteza de estar agindo corretamente e do medo de perder algo tão amado. Perder? Como? Não é nosso, recordam-se? Foi apenas um empréstimo.}''

A decisão de avançar com esta dissertação surgiu com duas ideias em mente: adquirir mais conhecimento e (tentar) ser um exemplo e inspiração para a então única filha, Ana Ramalhadeiro, mostrando que, com esforço, dedicação e devoção, a glória acaba por chegar. E não esquecer: nunca é tarde demais para aprender. Pelo caminho nasceu a Maria Ramalhadeiro, que, por agora, só se interessa pela ``Patrulha Pata'' e os ``Mistérios dos Bichos'', mas cá estarei para, a seu tempo, lhe mostrar o exemplo e servir de inspiração.

À mãe mais linda do mundo — Elsa Mariana Lopes (e) Mota — obrigado pela paciência, apoio e compreensão durante todo este processo. Sem ti, nada disto seria possível, até porque estavas responsável pelo pagamento das propinas.

Ao meu pai e irmão, José e Eng.\textsuperscript{o} Rui Ramalhadeiro - \textit{a.k.a.} GNUUU - pelo incentivo; aos meus amigos, que não me ajudaram em nada, a não ser chatear para jantares e lanches.

Um agradecimento especial à Professora Cristina Sá pela ajuda na elaboração do documento, ao Professor Ernesto Martins e ao Eng.\textsuperscript{o} José Garcia pelos conselhos e esclarecimentos técnicos.

\end{acknowledgements}
